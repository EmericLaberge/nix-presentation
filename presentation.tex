\documentclass{beamer}
\usetheme{metropolis}

\usepackage[utf8]{inputenc}
\usepackage[T1]{fontenc}
\usepackage[french]{babel}
\usepackage{listings}
\usepackage{graphicx}
\usepackage{hyperref}
\usepackage{xcolor}
\usepackage{svg}
\usepackage{minted}
% Define custom colors
\definecolor{commentgray}{rgb}{0.5, 0.5, 0.5}
\definecolor{keywordorange}{rgb}{0.8, 0.4, 0.0}
\definecolor{typeblue}{rgb}{0.2, 0.6, 1.0}
\definecolor{fggray}{rgb}{0.4, 0.4, 0.4}

\lstset{
    language=nix,
    basicstyle=\ttfamily\color{fggray},
    keywordstyle=\color{keywordorange}\bfseries,
    stringstyle=\color{green!60!black},
    commentstyle=\color{commentgray},
    morekeywords={with, let, in, rec, import, derivation},
    numbers=none,
    showstringspaces=false,
}
\title{Nix : Révolutionner la gestion des paquets}
\author{Emeric Laberge}
\date{\today}

\begin{document}

\maketitle

\begin{frame}{Plan de la présentation}
	\tableofcontents
\end{frame}

\section{Introduction}

\begin{frame}{Pourquoi Nix ?}
	\begin{itemize}
		\item Problèmes actuels de gestion des paquets :
		      \begin{itemize}
			      \item Conflits de dépendances
			      \item Environnements non reproductibles
		      \end{itemize}
		\item Importance pour la qualité logicielle :
		      \begin{itemize}
			      \item Fiabilité accrue
			      \item Maintenance facilitée
		      \end{itemize}
	\end{itemize}
	\footnotetext[1]{\href{https://nixos.org/}{Site officiel de Nix}}
\end{frame}

\section{Présentation de Nix}

\begin{frame}{Qu'est-ce que Nix ?}
	\begin{itemize}
		\item Gestionnaire de paquets fonctionnel
		\item Isolation complète des dépendances
		\item Environnements reproductibles
	\end{itemize}
	\begin{figure}
		\centering
		\includegraphics[width=0.3\linewidth]{nix-logo.pdf}
	\end{figure}
	\footnotetext[2]{\href{https://nixos.org/manual/nix/stable/}{Manuel de Nix}}
\end{frame}

\section{Fonctionnement Technique}

\begin{frame}[fragile]{Expressions Nix}
	\begin{itemize}
		\item Définition des paquets via des expressions pures
		\item Exemple d'une expression simple :
	\end{itemize}
	\rule{\linewidth}{0.4pt}

	\begin{lstlisting}[language=nix]
    # shell.nix
    {pkgs ? import <nixpkgs> {}}:
    pkgs.stdEnv.mkShell {
      buildInputs = with pkgs; [ ];
    }
  \end{lstlisting}
	\rule{\linewidth}{0.4pt}
	\footnotetext[3]{\href{https://nixos.org/manual/nixpkgs/stable/}{Guide Nixpkgs}}
\end{frame}

\begin{frame}[fragile]{Installation de Paquets}
	\begin{itemize}
		\item Installation isolée :
	\end{itemize}
	\begin{lstlisting}[language=bash]
$ nix-env -iA nixpkgs.hello
    \end{lstlisting}
	\begin{itemize}
		\item Aucun impact sur les autres paquets
		\item Désinstallation propre
	\end{itemize}
\end{frame}

\section{Point Original}

\begin{frame}{Nix et la Gestion Multi-Utilisateurs}
	\begin{itemize}
		\item Chaque utilisateur peut avoir son propre profil
		\item Partage efficace des ressources communes
		\item Sécurité renforcée
	\end{itemize}
	\footnotetext[4]{\href{https://nixos.org/manual/nix/stable/\#sec-multi-user-installation}{Installation multi-utilisateurs}}
\end{frame}

\section{Réflexion}

\begin{frame}{Avantages et Défis}
	\begin{itemize}
		\item \textbf{Avantages} :
		      \begin{itemize}
			      \item Environnements cohérents
			      \item Facilitation du déploiement continu
		      \end{itemize}
		\item \textbf{Défis} :
		      \begin{itemize}
			      \item Courbe d'apprentissage
			      \item Intégration avec les systèmes existants
		      \end{itemize}
	\end{itemize}
\end{frame}

\section{Conclusion}

\begin{frame}{Message Clé}
	\begin{block}{Adoptez Nix pour une Qualité Logicielle Optimale}
		\begin{itemize}
			\item Maîtrisez vos environnements
			\item Réduisez les bugs liés aux dépendances
			\item Accélérez vos cycles de développement
		\end{itemize}
	\end{block}
	\begin{center}
		\Large Merci de votre attention !
	\end{center}
\end{frame}

\begin{frame}{Questions}
	\begin{center}
		\Large Des questions ?
	\end{center}
\end{frame}

\appendix

\section{Sources}

\begin{frame}{Sources}
	\begin{enumerate}
		\item \href{https://nixos.org/}{Site officiel de Nix}
		\item \href{https://nixos.org/manual/nix/stable/}{Manuel de Nix}
		\item \href{https://nixos.org/manual/nixpkgs/stable/}{Guide Nixpkgs}
	\end{enumerate}
\end{frame}

\end{document}
